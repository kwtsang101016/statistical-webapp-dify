\documentclass[aspectratio=169]{beamer}
\usepackage[utf8]{inputenc}
\usepackage{graphicx}
\usepackage{listings}
\usepackage{xcolor}
\usepackage{tikz}
\usepackage{booktabs}
\usepackage{hyperref}

% Theme and colors
\usetheme{Madrid}
\usecolortheme{default}

% Custom colors
\definecolor{codegreen}{rgb}{0,0.6,0}
\definecolor{codegray}{rgb}{0.5,0.5,0.5}
\definecolor{codepurple}{rgb}{0.58,0,0.82}
\definecolor{backcolour}{rgb}{0.95,0.95,0.92}

% Code listing style
\lstdefinestyle{mystyle}{
    backgroundcolor=\color{backcolour},   
    commentstyle=\color{codegreen},
    keywordstyle=\color{blue},
    numberstyle=\tiny\color{codegray},
    stringstyle=\color{codepurple},
    basicstyle=\ttfamily\footnotesize,
    breakatwhitespace=false,         
    breaklines=true,                 
    captionpos=b,                    
    keepspaces=true,                 
    numbers=left,                    
    numbersep=5pt,                  
    showspaces=false,                
    showstringspaces=false,
    showtabs=false,                  
    tabsize=2
}

\lstset{style=mystyle}

% Title page information
\title[Statistical Webapp]{Building Your First Statistical Webapp}
\subtitle{STA2002: Probability and Statistics II}
\author{Fangda Song \& Ka Wai Tsang}
\institute[CUHK(SZ)]{School of Data Science, CUHK(SZ)}
\date{\today}

\begin{document}

% Title slide
\begin{frame}
\titlepage
\end{frame}

% Table of contents
\begin{frame}
\frametitle{Outline}
\tableofcontents
\end{frame}

\section{Introduction}

\begin{frame}
\frametitle{What We're Building}
\begin{columns}
\begin{column}{0.6\textwidth}
\begin{itemize}
\item \textbf{Interactive Statistical Webapp}
\item Distribution generator (Normal, Exponential, Binomial, Poisson, Uniform)
\item Real-time data visualization
\item Parameter estimation (MLE \& Method of Moments)
\item Summary statistics calculation
\item Educational content with formulas
\end{itemize}
\end{column}
\begin{column}{0.4\textwidth}
\begin{center}
\textbf{Modern UI} \\
\textbf{TypeScript} \\
\textbf{React + Tailwind}
\end{center}
\end{column}
\end{columns}
\end{frame}

\begin{frame}
\frametitle{Learning Objectives}
\begin{enumerate}
\item Understand modern web development with React and TypeScript
\item Implement statistical algorithms for data generation
\item Create interactive visualizations
\item Compare parameter estimation methods (MLE vs MoM)
\item Deploy a professional web application
\item Learn to troubleshoot common development issues
\end{enumerate}
\end{frame}

\section{Prerequisites \& Setup}

\begin{frame}
\frametitle{Prerequisites}
\begin{alertblock}{Important}
You don't need programming experience! We'll use AI code editors to help.
\end{alertblock}

\begin{columns}
\begin{column}{0.5\textwidth}
\textbf{Required:}
\begin{itemize}
\item Computer with internet
\item Node.js 18+ installed
\item Git installed
\item GitHub account
\item AI code editor (Cursor/VS Code)
\end{itemize}
\end{column}
\begin{column}{0.5\textwidth}
\textbf{Helpful:}
\begin{itemize}
\item Basic command line knowledge
\item Understanding of statistics concepts
\item Patience for troubleshooting
\end{itemize}
\end{column}
\end{columns}
\end{frame}

\begin{frame}[fragile]
\frametitle{Step 1: Install Node.js}
\begin{alertblock}{Critical Step}
Download and install Node.js from \href{https://nodejs.org}{nodejs.org}
\end{alertblock}

\begin{lstlisting}[language=bash, caption=Verify Installation]
# Check Node.js version
node --version
# Should show v18.x.x or higher

# Check npm version
npm --version
# Should show 9.x.x or higher
\end{lstlisting}

\begin{exampleblock}{Troubleshooting}
If commands not found:
\begin{itemize}
\item Restart your terminal/command prompt
\item Check if Node.js is in your PATH
\item Try reinstalling Node.js
\end{itemize}
\end{exampleblock}
\end{frame}

\begin{frame}[fragile]
\frametitle{Step 2: Create Project Directory}
\begin{lstlisting}[language=bash, caption=Create Project Folder]
# Navigate to your desired location
cd Desktop
# or wherever you want the project

# Create project folder
mkdir statistical-webapp-project
cd statistical-webapp-project
\end{lstlisting}

\begin{alertblock}{Important}
Use a simple path without spaces or special characters. Avoid paths like:
\begin{itemize}
\item \texttt{C:\textbackslash Users\textbackslash YourName\textbackslash MyDocuments}
\item \texttt{C:\textbackslash ProgramFiles\textbackslash Projects}
\end{itemize}
\end{alertblock}
\end{frame}

\section{Creating the React Project}

\begin{frame}[fragile]
\frametitle{Step 3: Create React Project}
\begin{alertblock}{PowerShell Users}
If you're using PowerShell and get interactive prompts:
\begin{itemize}
\item Type \texttt{y} and press Enter when asked "Ok to proceed?"
\item Select \textbf{React} when asked for framework
\item Select \textbf{TypeScript} when asked for variant
\end{itemize}
\end{alertblock}

\begin{lstlisting}[language=bash, caption=Create React App]
# Create React project with TypeScript
npm create vite@latest statistical-webapp -- --template react-ts

# Navigate to project
cd statistical-webapp

# Install dependencies
npm install
\end{lstlisting}
\end{frame}

\begin{frame}[fragile]
\frametitle{Step 4: Install Additional Packages}
\begin{lstlisting}[language=bash, caption=Install Required Packages]
# Install Tailwind CSS and dependencies
npm install -D tailwindcss@^3.4.0 postcss autoprefixer

# Install charting library
npm install recharts

# Install icons
npm install lucide-react

# Initialize Tailwind
npx tailwindcss init -p
\end{lstlisting}

\begin{alertblock}{Common Issues}
\begin{itemize}
\item If \texttt{npx} command fails, try: \texttt{npm install -g npx}
\item If permission errors, try: \texttt{npm install --legacy-peer-deps}
\end{itemize}
\end{alertblock}
\end{frame}

\section{Configuration Files}

\begin{frame}[fragile]
\frametitle{Step 5: Configure Tailwind CSS}
\begin{lstlisting}[language=bash, caption=Update tailwind.config.js]
/** @type {import('tailwindcss').Config} */
export default {
  content: [
    "./index.html",
    "./src/**/*.{js,ts,jsx,tsx}",
  ],
  theme: {
    extend: {
      colors: {
        primary: {
          50: '#eff6ff',
          500: '#3b82f6',
          600: '#2563eb',
          700: '#1d4ed8',
        }
      }
    },
  },
  plugins: [],
}
\end{lstlisting}
\end{frame}

\begin{frame}[fragile]
\frametitle{Step 6: Update CSS File}
\begin{lstlisting}[language=bash, caption=Update src/index.css]
@tailwind base;
@tailwind components;
@tailwind utilities;

@import url('https://fonts.googleapis.com/css2?family=Inter:wght@300;400;500;600;700&display=swap');

@layer base {
  html {
    font-family: 'Inter', system-ui, sans-serif;
  }
  
  body {
    @apply bg-gray-50 text-gray-900;
  }
}

@layer components {
  .btn-primary {
    @apply bg-primary-600 hover:bg-primary-700 text-white font-medium py-2 px-4 rounded-lg transition-colors duration-200;
  }
  
  .card {
    @apply bg-white rounded-xl shadow-sm border border-gray-200 p-6;
  }
}
\end{lstlisting}
\end{frame}

\section{Running the Application}

\begin{frame}[fragile]
\frametitle{Step 7: Start Development Server}
\begin{lstlisting}[language=bash, caption=Start the App]
# Start development server
npm run dev
\end{lstlisting}

\begin{alertblock}{Expected Output}
You should see:
\begin{lstlisting}[language=bash]
> statistical-webapp@0.0.0 dev
> vite

  VITE v7.1.5  ready in 675 ms

  Local:   http://localhost:5173/
  Network: use --host to expose
\end{lstlisting}
\end{alertblock}

\begin{exampleblock}{Success!}
Open your browser and go to: \texttt{http://localhost:5173/}
You should see the default React app.
\end{exampleblock}
\end{frame}

\begin{frame}
\frametitle{Common Issues \& Solutions}
\begin{columns}
\begin{column}{0.5\textwidth}
\textbf{Port Already in Use:}
\begin{itemize}
\item Try: \texttt{npm run dev -- --port 3000}
\item Or kill the process using port 5173
\end{itemize}

\textbf{Module Import Errors:}
\begin{itemize}
\item Clear browser cache (Ctrl+F5)
\item Restart development server
\item Check file paths and exports
\end{itemize}
\end{column}
\begin{column}{0.5\textwidth}
\textbf{Tailwind Not Working:}
\begin{itemize}
\item Check \texttt{tailwind.config.js}
\item Verify CSS imports
\item Restart development server
\end{itemize}

\textbf{PowerShell Issues:}
\begin{itemize}
\item Use Command Prompt instead
\item Or enable QuickEdit mode
\item Avoid complex paths
\end{itemize}
\end{column}
\end{columns}
\end{frame}

\section{Copying the Application Code}

\begin{frame}
\frametitle{Step 8: Get the Application Code}
\begin{alertblock}{Important}
Don't try to write the code from scratch! We'll provide the complete working code.
\end{alertblock}

\begin{enumerate}
\item Download the provided \texttt{App.tsx} file
\item Replace the existing \texttt{src/App.tsx} with the new code
\item Save the file
\item Refresh your browser
\end{enumerate}

\begin{exampleblock}{What You'll See}
\begin{itemize}
\item Beautiful statistical webapp interface
\item Distribution selection panel
\item Parameter controls with sliders
\item Data generation functionality
\item Summary statistics display
\end{itemize}
\end{exampleblock}
\end{frame}

\begin{frame}[fragile]
\frametitle{Step 9: Test the Application}
\begin{enumerate}
\item Select "Normal Distribution"
\item Adjust mean and standard deviation sliders
\item Set sample size to 1000
\item Click "Generate Data"
\item Observe the results!
\end{enumerate}

\begin{alertblock}{Expected Behavior}
\begin{itemize}
\item Sliders should update values in real-time
\item "Generate Data" button should create new data
\item Statistics should appear in the right panel
\item No console errors in browser developer tools
\end{itemize}
\end{alertblock}
\end{frame}

\section{Deployment}

\begin{frame}[fragile]
\frametitle{Step 10: Deploy to GitHub Pages}
\begin{lstlisting}[language=bash, caption=Initialize Git Repository]
# Initialize git repository
git init

# Add all files
git add .

# Create initial commit
git commit -m "Initial statistical webapp"

# Add remote repository (replace with your GitHub repo)
git remote add origin https://github.com/yourusername/statistical-webapp.git

# Push to GitHub
git push -u origin main
\end{lstlisting}
\end{frame}

\begin{frame}[fragile]
\frametitle{Step 11: Configure GitHub Pages}
\begin{enumerate}
\item Go to your GitHub repository
\item Click "Settings" tab
\item Scroll to "Pages" section
\item Select "GitHub Actions" as source
\item The provided workflow will automatically deploy your app
\end{enumerate}

\begin{alertblock}{Deployment Workflow}
The repository includes a \texttt{.github/workflows/deploy.yml} file that:
\begin{itemize}
\item Builds your React app
\item Deploys to GitHub Pages
\item Updates automatically on every push
\end{itemize}
\end{alertblock}
\end{frame}

\section{Troubleshooting Guide}

\begin{frame}
\frametitle{Common Problems \& Solutions}
\begin{columns}
\begin{column}{0.5\textwidth}
\textbf{Import Errors:}
\begin{itemize}
\item Check file paths
\item Verify exports exist
\item Restart development server
\item Clear browser cache
\end{itemize}

\textbf{Tailwind Issues:}
\begin{itemize}
\item Use Tailwind v3.4.0 (not v4)
\item Check config file syntax
\item Verify CSS imports
\end{itemize}
\end{column}
\begin{column}{0.5\textwidth}
\textbf{PowerShell Problems:}
\begin{itemize}
\item Use Command Prompt
\item Enable QuickEdit mode
\item Avoid spaces in paths
\end{itemize}

\textbf{Port Issues:}
\begin{itemize}
\item Try different port
\item Kill existing processes
\item Check firewall settings
\end{itemize}
\end{column}
\end{columns}
\end{frame}

\begin{frame}
\frametitle{Getting Help}
\begin{columns}
\begin{column}{0.5\textwidth}
\textbf{AI Code Editor:}
\begin{itemize}
\item Ask AI to explain errors
\item Request code fixes
\item Get step-by-step guidance
\end{itemize}
\end{column}
\begin{column}{0.5\textwidth}
\textbf{Resources:}
\begin{itemize}
\item React documentation
\item Tailwind CSS docs
\item Vite documentation
\item Stack Overflow
\end{itemize}
\end{column}
\end{columns}

\begin{alertblock}{Pro Tips}
\begin{itemize}
\item Read error messages carefully
\item Check browser developer console
\item Test incrementally (don't change everything at once)
\item Use version control (git) to track changes
\end{itemize}
\end{alertblock}
\end{frame}

\section{Next Steps}

\begin{frame}
\frametitle{What's Next?}
\begin{enumerate}
\item \textbf{Customize the UI}: Change colors, fonts, layout
\item \textbf{Add New Distributions}: Chi-square, t-distribution, F-distribution
\item \textbf{Enhance Visualizations}: Add box plots, Q-Q plots
\item \textbf{Implement Confidence Intervals}: Add CI calculations
\item \textbf{Add Hypothesis Testing}: t-tests, chi-square tests
\item \textbf{Create Data Export}: Download data as CSV/JSON
\end{enumerate}

\begin{exampleblock}{Future Enhancements}
As you learn more statistics, you can add:
\begin{itemize}
\item ANOVA analysis
\item Linear regression
\item Bootstrap methods
\item Central Limit Theorem demonstrations
\end{itemize}
\end{exampleblock}
\end{frame}

\begin{frame}
\frametitle{Assessment \& Submission}
\begin{columns}
\begin{column}{0.5\textwidth}
\textbf{What to Submit:}
\begin{itemize}
\item GitHub repository URL
\item Live deployed website
\item Brief README explaining features
\item Screenshots of your app
\end{itemize}
\end{column}
\begin{column}{0.5\textwidth}
\textbf{Evaluation Criteria:}
\begin{itemize}
\item Functionality (does it work?)
\item UI/UX design
\item Code organization
\item Educational value
\item Creativity \& extensions
\end{itemize}
\end{column}
\end{columns}

\begin{alertblock}{Deadline}
Submit your GitHub repository URL and deployed website link by [DATE].
\end{alertblock}
\end{frame}

\begin{frame}
\frametitle{Questions \& Support}
\begin{center}
\Large \textbf{Questions?}

\vspace{1em}

\textbf{Office Hours:}
\begin{itemize}
\item Fangda Song: Tue 3:30-4:30 pm, Rm 420d
\item Ka Wai Tsang: Mon 10:30-11:30 am, Rm 505b
\end{itemize}

\vspace{1em}

\textbf{Teaching Assistants:}
\begin{itemize}
\item Ruicong Wang: Mon 10:00-11:00 am
\item Wendi Ren: Thu 4:00-5:00 pm
\item Bokun Yu: Thu 4:00-5:00 pm
\end{itemize}

\vspace{1em}

\textbf{Online Support:}
Tencent Meeting: 748-5967-3028
\end{center}
\end{frame}

\begin{frame}
\begin{center}
\Huge \textbf{Good Luck!}

\vspace{1em}

\Large Build something amazing!

\vspace{2em}

\textbf{Remember:} Start simple, test often, ask for help!
\end{center}
\end{frame}

\end{document}

